\rm{Reg4_{DAPI + Bodipy + CD45}} \; = \; & -0.25 * \frac{\Sigma_{CD45} - 68.47\;  dBct }{2.76 \;  dBct } \\+ & \; 0.16 * \frac{\Sigma_{Bodipy} - 73.60 \;  dBct }{5.05 \;  dBct  } \\+ & \; 0.12 * \frac{<M>_{Bodipy} - 4.29 \; }{1.00 \;  } \\+ & \; 0.05 * \frac{\Sigma_{DAPI} - 74.76 \;  dBct }{3.09 \;  dBct  } \\- & \; 0.01 * \frac{<r_f>_{CD45} - 0.98 \;  \mu m^{-1}  }{0.03 \;  \mu m^{-1}  }